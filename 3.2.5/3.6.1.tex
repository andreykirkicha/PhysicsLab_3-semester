\documentclass[14pt,a4paper]{article}
\usepackage[14pt]{extsizes}
\usepackage[left=1.5cm, right=1.5cm, top=1.5cm, bottom=1.5cm]{geometry}
\usepackage[utf8]{inputenc}
\usepackage[T2A]{fontenc}
\usepackage[english, russian]{babel}
\usepackage{amsmath,amsfonts,amssymb,amsthm,mathtools} 
\usepackage{amsfonts}
\usepackage{amssymb}
\usepackage{titleps}
\usepackage{hyperref}
\usepackage{float}
\usepackage{graphicx}
\usepackage{multirow}
\usepackage{hhline}
\usepackage{wrapfig}
\usepackage{tikz}
\usepackage{pgfplots}
\usepackage{xcolor}
\usepackage{subfig}
\usepackage{upgreek}

\newcommand{\w}[1]{\text{#1}}
\newcommand{\und}[1]{\underline{#1}}
\newcommand{\img}[3]{
	\begin{figure}[H]
	\begin{center}
	\includegraphics[scale=#2]{#1}
	\end{center}
	\begin{center}
 	\textit{#3}
	\end{center}
	\end{figure}
}
\newcommand{\aw}[1]{
	\begin{center}
	\textit{#1}
	\end{center}
	\n
}
\newcommand{\be}[1]{
	\begin{center}
	\boxed{#1}
	\end{center}
}
\newcommand{\beb}[1]{
	\begin{equation}
	\boxed{#1}
	\end{equation}
}
\newcommand{\eb}[1]{
	\begin{equation}
	#1
	\end{equation}
}
\newcommand{\n}{\hfill \break}
\newcommand{\x}{\cdot}
\begin{document}

\section*{Работа 3.7.1}	
	\section*{Скин-эффект}
    \section*{Киркича Андрей, Б01-202, МФТИ}

\textbf{В работе используются}: генератор сигналов АКИП–3420, соленоид, намотанный на полый цилиндрический каркас, медный экран в виде полого цилиндра, измерительная катушка, амперметр, вольтметр, двухканальный осциллограф GOS–620, RLC-метр.

\section*{Теоретические сведения}

\ \ \ \ Толщина скин-слоя проводника:
\begin{equation}
    \delta = \sqrt{\frac{2}{\omega\sigma\mu\mu_0}}.
    \label{eq:delta}
\end{equation}

Связь полей внутри и снаружи цилиндра:
\begin{equation}
    H_1 = \frac{H_0}{\ch(\alpha h) + \frac{1}{2} \alpha a \sh(\alpha h)} 
    \text{;\ \ \ }
    \alpha = \sqrt{i\omega \sigma \mu_0} = \frac{\sqrt{2}}{\delta}e^{i\pi/4}.
    \label{eq:svyaz_poley}
\end{equation}

Отношение амплитуд полей:
\begin{equation}
    \frac{|H_1|}{|H_0|} = c \cdot \frac{U}{\nu I} = c \xi.
    \label{eq:otnoshenie_amplitud}
\end{equation}

\begin{figure}[h]
    \centering
    \includegraphics[width=17cm, height=8cm]{Pictures/scheme.jpg}
    \label{fig:scheme}
\end{figure}

\section*{Результаты измерений}

Перед началом работы запишем данные установки и вычислим необходимую нам для измерений частоту $\nu_h$ по формуле (1), приняв $\delta = h$.

\begin{table}[!ht]
\centering
    \begin{tabular}{|r|r|r|r|r|r|}
    \hline
        $d_{\text{нар}}, \ \text{мм}$  & $d_{\text{стен}}, \ \text{мм}$ & $\sigma, \  \frac{\text{См}}{\text{м}}$ & $\nu_{\text{h}}, \ \text{кГц} $ & $\nu_0, \ \text{Гц}$ & $\text{A}, \ \text{В}$ \\ \hline
        45 & 1,5 & 0,08 & 2,2 & 22 & 8 \\ \hline
    \end{tabular}
\end{table}
Далее приступим к измерению проводимости разными методами.
\subsection*{Измерение проводимости через отношение амплитуд}
Снимем зависимость тока через амперметр и напряжения на вольтметре от частоты, выставляемой на генераторе. Подсчитаем $\xi$, руководствуясь формулой (3).

\begin{table}[!ht]
    \centering
    \begin{tabular}{|c|r|r|r|r|r|r|r|r|r|r|}
    \hline
        $\nu, \text{Гц}$ & 22 & 30 & 40 & 50 & 60 & 70 & 80 & 90 & 100 & 110 \\ \hline
        $I, \text{А}$ & 0,53 & 0,53 & 0,52 & 0,51 & 0,51 & 0,40 & 0,49 & 0,48 & 0,47 & 0,46 \\ \hline
        $U, \text{В}$ & 0,16 & 0,22 & 0,29 & 0,36 & 0,42 & 0,47 & 0,51 & 0,55 & 0,59 & 0,62 \\ \hline
        $\xi \cdot 10^{-2}$ & 1,44 & 1,43 & 1,41 & 1,39 & 1,37 & 1,34 & 1,31 & 1,28 & 1,25 & 1,22 \\ \hline
    \end{tabular}
\end{table}

В области частот $\nu \ll \nu_h$ $\alpha h \ll 1$. Из формулы (2) получаем:

\[ (c\xi)^2 \approx \frac{1}{1+A\nu^2} \quad \Leftrightarrow \quad  \frac{1}{\xi^2}=B\nu^2 + c^2, \text{ где } B=\pi a h \sigma \mu_0 c. \]

\begin{figure}[h]
    \centering
    \includegraphics[width=18cm, height=8cm]{Pictures/graph1.jpg}
    \label{fig:scheme}
\end{figure}

Рассчитанное значение индуктивности: $B = (1,7\pm0,3) \x 10^{-1} \ 1/\text{Гц}^2$. Значит, $\sigma = (4,2\pm0,7) \cdot 10^7 \ \text{См/м}$ и $c = 69 \pm 10$.  

\subsection*{Измерение проводимости через разность фаз на низких частотах}
Измерим ток и напряжение в зависимости от частоты, параллельно считывая с осциллографа величину фазового сдвига $\psi$.

\begin{table}[!ht]
    \centering
    \resizebox{\textwidth}{!}{
    \begin{tabular}{|r|r|r|r|r|r|r|r|r|r|r|r|r|r|r|r|}
    \hline
        $\nu, \text{Гц}$ & 110 & 130 & 150 & 170 & 190 & 210 & 220 & 330 & 440 & 550 & 660 & 770 & 880 & 990 & 1100 \\ \hline
        $I, \text{мА}$ & 462 & 432 & 421 & 411 & 403 & 395 & 392 & 369 & 355 & 344 & 335 & 326 & 317 & 308 & 299 \\ \hline
        $U, \text{В}$ & 620 & 650 & 689 & 720 & 743 & 761 & 768 & 811 & 819 & 814 & 802 & 787 & 768 & 749 & 730 \\ \hline
        $\xi \cdot 10^{-2}$ & 1,22 & 1,16 & 1,09 & 1,03 & 0,97 & 0,92 & 0,89 & 0,67 & 0,52 & 0,43 & 0,36 & 0,31 & 0,28 & 0,25 & 0,22 \\ \hline
        $\phi, \ \text{рад}$ & 0,98 & 0,81 & 0,79 & 0,63 & 0,64 & 0,63 & 0,72 & 0,39 & 0,29 & 0,22 & 0,16 & 0,13 & 0,11 & 0,04 & 0,00 \\ \hline
        $\psi, \ \text{рад}$ & -0,59 & -0,76 & -0,79 & -0,94 & -0,93 & -0,94 & -0,85 & -1,19 & -1,28 & -1,35 & -1,41 & -1,45 & -1,47 & -1,53 & -1,57 \\ \hline
    \end{tabular}
    }
\end{table}

На основе данных из таблицы строим график на низких частотах.

\begin{figure}[h]
    \centering
    \includegraphics[width=18cm, height=9cm]{Pictures/graph2.jpg}
    \label{fig:scheme}
\end{figure}

Согласно формуле $tg \psi = \pi a h \sigma \mu_0 \nu$ получаем, что $\sigma = (8,3\pm1,3) \cdot 10^7$ См/м.

\subsection*{Измерение проводимости через разность фаз на высоких частотах}
Повторим измерения для более высоких частот.

\begin{table}[!ht]
    \centering
    \resizebox{\textwidth}{!}{
    \begin{tabular}{|r|r|r|r|r|r|r|r|r|r|r|r|r|r|r|r|}
    \hline
        $\nu, \text{Гц}$ & 1100 & 1300 & 1700 & 2200 & 2800 & 3500 & 4400 & 5500 & 7000 & 8700 & 11000 & 14000 & 17000 & 22000 & 28000 \\ \hline
        $I, \text{мА}$ & 299 & 277 & 249 & 219 & 187 & 157 & 129 & 105 & 84 & 67 & 52 & 39 & 28 & 18 & 9 \\ \hline
        $U, \text{В}$ & 730	& 676 & 609 & 532 & 452 & 374 & 303 & 239 & 184 & 140 & 104 & 76 & 57 & 48 & 46 \\ \hline
        $\xi \cdot 10^{-3}$ & 2,22 & 1,76 & 1,40 & 1,11 & 0,88 & 0,69 & 0,54 & 0,41 & 0,32 & 0,24 & 0,18 & 0,14 & 0,12 & 0,12 & 0,19 \\ \hline
        $\phi, \ \text{рад}$ & 0,00 & 0,09 & 0,14 & 0,26 & 0,39 & 0,50 & 0,86 & 1,05 & 1,41 & 1,42 & 1,48 & 1,57 & 1,69 & 1,99 & 2,35 \\ \hline
        $\psi, \ \text{рад}$ & -1,57 & -1,48 & -1,43 & -1,32 & -1,18 & -1,07 & -0,71 & -0,52 & -0,16 & -0,15 & -0,09 & 0,00 & 0,12 & 0,42 & 0,79 \\ \hline
    \end{tabular}
    }
\end{table}

\begin{figure}[H]
    \centering
    \includegraphics[width=18cm, height=9cm]{Pictures/graph3.jpg}
    \label{fig:scheme}
\end{figure}

При $\delta \ll h$ выполняется:
\begin{equation*}
    \psi - \pi/4 = k\cdot \sqrt{\nu}; \quad \ k = h\sqrt{\pi\mu_0\sigma}.
\end{equation*}
Таким образом, $k =  (1,8 \pm 0,3) \cdot 10^{-2} \ \text{рад/Гц}$, тогда $\sigma = (3,45 \pm 0,15) \cdot 10^7$ См/м.

\section*{Заключение}
В ходе выполнения работы мы проверили формулы для вычисления параметров скин-эффекта в соленоидальной катушке, вычислив отношения магнитных полей как на малых частотах токов, проходящих через катушку, так и на больших.

\section*{Литература}
\noindent
1. \textit{Сивухин Д. В.} Общий курс физики. Учеб. пособие: Для вузов. Т. III. Электричество. - 6-е издание. М.: ФИЗМАТЛИТ, 2019 \\
2. \textit{Никулин М.Г., Попов П.В., Нозик А.А., и др.} Лабораторный практикум по общей физике: учеб. пособие. В трёх томах. Т. II. Электричество и магнетизм. - 2-е издание. М.: МФТИ, 2019

\end{document}

\begin{document}
\section*{Работа 3.2.5}	
	\section*{Свободные и вынужденные колебания в электрическом контуре}
	\subsection*{Андрей Киркича, Б01-202, МФТИ}
	\n
	\textbf{В работе используются: }
осциллограф; генератор сигналов специальной формы; магазин сопротивления; магазин емкости; магазин индуктивности; соединительная коробка с шунтирующей емкостью; соединительные одножильные и коаксиальные провода.

\section*{Экспериментальная установка}
Картина колебаний напряжения на емкости наблюдается на экране двухканального осциллографа. Для возбуждения затухающих колебаний используется генератор сигналов специальной
формы. Сигнал с генератора поступает через конденсатор $C_1$ на вход колебательного контура. Данная емкость необходима, чтобы выходной импеданс генератора был много меньше импеданса колебательного контура и не влиял на процессы, проходящие в контуре.

\begin{figure}[H]
\centering
\subfloat[Схема установки для исследования вынужденных колебаний]{\includegraphics[width=0.45\textwidth]{1.png}}
\qquad
\subfloat[Схема установки для исследования АЧХ и ФЧХ]{\includegraphics[width=0.45\textwidth]{2.png}}
\end{figure}
\n
При изучении свободно затухающих колебаний генератор специальных сигналов на вход колебательного контура подает периодические короткие импульсы, которые заряжают конденсатор $C$. За время между
последовательными импульсами происходит разрядка конденсатора через резистор и катушку индуктивности. Напряжение на конденсаторе $U_C$ поступает на вход канала 1(X) электронного осциллографа. Для наблюдения фазовой картины затухающих колебаний на канал 2(Y) подается напряжение с резистора $R$ (пунктирная линия на
схеме установки), которое пропорционально току $I$.
\n\n
При изучении возбужденных колебаний на вход колебательного контура подается синусоидальный сигнал. С помощью осциллографа возможно измерить зависимость амплитуды возбужденных колебаний в зависимости от частоты внешнего сигнала, из которого возможно определить добротность колебательного контура. Альтернативным способом расчета добротности контура является определение декремента затухания по картине установления возбужденных колебаний. В этом случае генератор сигналов используется для подачи цугов синусоидальной формы.

\section*{Выполнение работы}
\subsection*{Измерение периодов свободных колебаний}

На генераторе предварительно была установлена последовательность импульсов со следующими параметрами: длительность импульсов - 10 мкс, частота повторения - 100 Гц, амплитуда сигнала - 20 В. На магазине сопротивлений было установлено минимальное значение, на магазине индуктивностей - $L = 100$ мГн (во время проведения работы это значение оставалось постоянным), на магазине емкостей - $C = 0$ мкФ. На экране осциллографа была получена картина свободных затухающих колебаний.
\n\n
Был измерен период затухающих колебаний $T = (67,0 \pm 1,0)$ мкс. По этому значению была найдена нулевая ёмкость колебательного контура:
\[ C_0 = \frac{T^2}{4 \pi^2 L} = (1,1 \pm 0,2) \text{ нФ.} \]
Изменяя ёмкость (по курбелям) от 0 до 9 нФ, мы провели измерения периодов:
\begin{table}[H]
\centering
\begin{tabular}{|r|r|r|r|r|r|}
\hline
$C$, нФ & 1 & 3 & 5 & 7 & 9 \\ \hline
$T_{\text{эксп}}$, мкс & $110 \pm 9$ & $127 \pm 9$ & $142 \pm 9$ & $154 \pm 9$ & $169 \pm 9$ \\ \hline
$T_{\text{теор}}$, мкс & $108 \pm 7$ & $126 \pm 7$ & $141 \pm 7$ & $153 \pm 7$ & $166 \pm 7$ \\ \hline
\end{tabular}
\end{table}
\n
Учитывая, что $R_{\Sigma} = R + R_{L}$, построим график:
\begin{center}
\includegraphics[scale=2]{6.jpg}
\end{center}
\n
Как видно из графика, зависимость является линейной, что подтверждается теоретической справкой.
\subsection*{Критическое сопротивление и декремент затухания}
Ёмкость $C^*$, при которой собственная частота колебаний составляет $\nu_0 = 6,5$ кГц:
\[ C^* = \frac{1}{4 \pi^2 L \nu_0^2} = (6,1 \pm 0,6) \text{ нФ,} \]
и критическое сопротивление контура:
\[ R_{\text{крит}} = 2\sqrt{\frac{L}{C^*}} = (7,5 \pm 0,3) \text{ кОм.} \]
На магазине емкостей было установлено значение $0,007$ мкФ, близкое к $C^*$. Увеличивая сопротивление $R$ от $0$ до $R_{\text{крит}}$, мы определили сопротивление, при котором колебательный режим переходит в апериодический: $R_{\text{апер}} = 4$ кОм.
\begin{wrapfigure}{r}{0.5\textwidth}
\includegraphics[scale=3]{tab1.png}
\vspace{-2cm}
\end{wrapfigure}
Затем мы устанавливали сопротивления в интервале $(0,05 - 0,25)R_{\text{крит}}$ и для каждого значения рассчитывали логарифмический декремент затухания по формуле: $\Theta = \frac{1}{n}\ln \frac{U_{m}}{U_{n + m}}$, где $n$ - целое число периодов, разделяющее максимумы.
\n\n
Для максимального и минимального значений $\Theta$ были найдены добротности:
\[ Q_{min} = \pi / \Theta_{min} = 11,22 \pm 0,07, \qquad \qquad Q_{max} = \pi / \Theta_{max} = 2,49 \pm 0,09. \]
\subsection*{Свободные колебания на фазовой плоскости}
\begin{wrapfigure}{l}{0.5\textwidth}
\vspace{-0.5cm}
\includegraphics[scale=0.2]{3.jpg}
\vspace{-40pt}
\end{wrapfigure}
На магазине сопротивлений было установлено значение $R = 0,3$ кОм, на канал 2(Y) осциллографа подана величина падения напряжения с резистора. Переключив осциллограф в режим XY, мы получили картину затухающих колебаний на фазовой плоскости. Она представляет из себя спираль, сходящуюся к центру.
\n
\subsection*{Резонансные кривые}
В этом пункте работы мы установили $C = C^*$, $R = 0,3$ кОм, подали сигнал с генератора одновременно на колебательный контур и канал 2 осциллографа. При частотах, близких к резонансным, наблюдался устойчивый синусоидальный сигнал, амплитуда колебаний при этом стремилась к максимуму. Была найдена резонансная частота: $\nu_{\text{рез}} = 6,5$ кГц.
\n\n
АЧХ и ФЧХ контура:
\begin{figure}[H]
\centering
\subfloat[Амплитудно-частотная характеристика]{\includegraphics[width=0.45\textwidth]{5.jpg}}
\qquad
\subfloat[Фазово-частотная характеристика]{\includegraphics[width=0.45\textwidth]{7.jpg}}
\end{figure}
\section*{Заключение}
По данным, полученным в ходе экспериментов, были рассчитаны добротности контура при разных значениях сопротивления, получена картина затухающих колебаний на фазовой плоскости, а также построены амплитудно-частотная и фазово-частотная характеристики.

\section*{Литература}
\n
1. Никулин М.Г., Попов П.В., Нозик А.А., и др. Лабораторный практикум по общей физике: учеб. пособие. В трёх томах. Т. II. Электричество и магнетизм. - 2-е издание. М.: МФТИ, 2019\n
2. Сивухин Д. В. Общий курс физики. Учеб. пособие: Для вузов. Т. III. Электричество. - 6-е издание. М.: ФИЗМАТЛИТ, 2019
\end{document}